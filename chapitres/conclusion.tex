\section{Conclusion}
	\subsection{Conclusion générale}
		\vspace{0.3cm}

		Le projet fût très intéressant. Il nous a permis de nous pencher sur des cas réseau que nous abordons peu en cours. L'éventualité d'une mise en production des résultats de nos travaux nous a incités à plus d'exigences sur la qualité de ce projet. L'utilisation de l'outil Zabbix nous a beaucoup ralentis du fait d’un manque de documentation et de sa précision. Nous avons donc créé un sujet sur le forum pour les difficultés rencontrées quant à l'obtention de la valeur du 95\up{ème} centile dans une variable ou une macro. Seulement, peu de réactivité de la part de la communauté. En effet, après plusieurs semaines, nous avons observé peu de vues (100-200) et personne n'a pu nous aider.\\

		Nous sommes déçus de n'avoir pu obtenir de meilleurs résultats. Néanmoins, ils pourront servir à l'association pour une conclusion finale de l'outil Zabbix et de ses limites. Nous aurions souhaité avoir plus de temps pour pouvoir effectuer des recherches supplémentaires et obtenir des résultats concluants et directement utilisables en production.\\

		Le peu de ressources matérielles en classe nous a aussi beaucoup ralentis. Par exemple, il nous a fallu attendre plusieurs jours pour obtenir l'installation d'une deuxième carte réseau sur nos deux serveurs. Nous étions donc freinés dans l’avancement du projet, n’ayant plus notre environnement de test. L'association nous a donc mis à disposition du matériel : un câble croisé et un commutateur, éléments manquants en salle bloquants pour relier nos serveurs en direct et ne pas être parasités.\\

	\subsection{Expérience acquise}
		\vspace{0.3cm}

		Le travail avec l'association LDN nous a permis de comprendre comment fonctionnait un FAI associatif sur un plan technique mais aussi commercial. Nous ignorions jusqu'alors la contrainte du transit, peering et la facturation au 95\up{ème} centile.\\

		La reproduction du réseau a été très bénéfique pour l'acquisition de nombreuses compétences, sur les configurations réseau d'un système Debian. Elle nous a permis une meilleure compréhension et nous nous sommes orientés vers un changement de nos méthodes de travail. L'utilisation des outils comme \verb?ip? ou \verb?tc? était jusqu'à présent inconnue bien qu'utilisée en entreprise. Le travail avec un FAI associatif nous a fait comprendre la réelle pénurie d'adresses IPv4 et des contraintes qu'elle génère ; ainsi nous avons pu effectuer notre première expérience avec les adresses IPv6.\\

		Il a été fort agréable de découvrir un nouvel outil de monitoring comme Zabbix, qui reste très puissant malgré le fait qu'il ne peut pas répondre totalement à nos attentes dans le cadre du projet.\\

	\subsection{Remerciements}
		\vspace{0.3cm}

		Nous tenons à remercier nos deux tuteurs, Sébastien JEAN et Julien VAUBOURG, pour le temps qu'ils nous ont dédié, pour nous avoir assistés lors des difficultés rencontrées, leur présence hebdomadaire, leur réactivité pour nous répondre, ainsi que pour la proposition du projet. Nous remercions aussi l'association LDN pour le prêt du matériel et l'accès à leur base de connaissances via le wiki.

\newpage
