\section{Lorraine Data Network}
	\subsection{Qu'est-ce que LDN et qui sont-ils ?}
		\vspace{0.3cm}

		\begin{wrapfigure}[10]{r}{6.5cm}
			\vspace{-0.7cm}
			\includegraphics[width=6.5cm]{logoLdn1.png}
		\end{wrapfigure}
	
		\textbf{L}orraine \textbf{D}ata \textbf{N}etwork (\textbf{LDN}) est une association de type loi 1901. Elle n’a donc aucun client, uniquement des adhérents, auxquels certains services peuvent être proposés (un accès ADSL, par exemple), contre compensation financière. Les statuts qui définissent l’association sont ainsi : \emph{«L’association a pour but : la promotion, l’utilisation et le développement des réseaux Internet dans le respect de leur éthique en favorisant en particulier les utilisations à des fins de recherche ou d’éducation sans volonté commerciale.»}\\

		En n'ayant aucune volonté commerciale, Lorraine Data Network propose une alternative, au modèle économique actuel des FAI classiques, en s’engageant à respecter les libertés fondamentales de l'Internet. En s’invitant dans des débats où seuls les principaux acteurs étaient jusqu’alors conviés, LDN milite aux côtés des autres associations (ALTERN, La Quadrature du Net, FDN, etc.) pour défendre nos libertés sur Internet. En s’invitant dans un maximum d’événements de la région, LDN a pour but de sensibiliser la population lorraine aux enjeux de la neutralité des réseaux et des conséquences de son abandon progressif.\\

		Actuellement, l'association compte une soixantaine de passionnés et bénévoles qui prennent sur leur temps libre pour faire avancer l’association. Ces derniers sont aussi en majorité des passionnés d’Internet et d’informatique, mais pas uniquement. L'association compte énormément de soutiens de personnes qui n’ont aucun lien direct avec l’informatique, mais qui sont soucieuses de leurs libertés. Ils sont donc des informaticiens, des étudiants, des professeurs, etc. De tout âge et de tout horizon, mais partageant le même idéal.\\

	\subsection{LDN face à la loi}
		\vspace{0.3cm}

		LDN n'accepte que les requêtes judiciaires, qui passent par les procédures telles qu'elles sont écrites dans la loi, sans aucune sorte de compromis possible. Ce strict respect des règles serait rarement une politique suivie par les gros opérateurs, qui auraient plutôt tendance à livrer des informations sur un simple coup de téléphone des autorités. \\
		
		Quelles que soient les obligations légales auxquelles l'association est soumise en tant que fournisseur d’accès à Internet, elle ne peut garantir à terme qu’une totale transparence vis-à-vis des adhérents ainsi qu’une mise au vote de toutes les décisions qui peuvent être soumises à contestation. Si ces obligations posent un problème vis-à-vis de l’éthique ou de la neutralité ou des libertés du Net, LDN fera son possible pour les combattre, à vos côtés et aux côtés des autres entités militantes.\\

	\subsection{Les services proposés}
		\vspace{0.3cm}
		LDN propose de nombreux services. Bien entendu en tant que FAI, l'association propose une connexion à l'Internet en fournissant l'accès à une dizaine de lignes ADSL. Certains services comme l'hébergement ou bien la fibre ne sont pas encore opérationnels mais sont en cours de développement. Voici globalement la liste des services disponibles : \\

		\begin{itemize}
			\item[$\bullet$] \textbf{VPN}, permet d’établir un tunnel chiffré entre votre ordinateur/téléphone/tablette et un serveur de LDN. Ainsi, quelle que soit la connexion Internet utilisée, vous récupérez vos IP habituelles et vous pouvez profiter d’un accès à Internet neutre, non filtré et sécurisé. Livré avec un bloc IPv6 /48 et une adresse IPv4 fixes.

			\item[$\bullet$] \textbf{Espace web mutualisé}, vous envoyez vos fichiers sur votre espace de stockage, et votre site web est directement accessible. Par exemple, avec des solutions comme WordPress, ce service est idéal pour proposer un blog en ayant peu de connaissances en informatique. Espace accessible en SFTP/FTPS/FTP avec une base de données MySQL et un vhost Apache avec PHP5. Espace de départ de 2Go. Utilisation d’un ou plusieurs noms de domaines personnels ou d’un ou plusieurs sous-domaines gratuits de acteurdu.net ou altu.fr.

			\item[$\bullet$] \textbf{Pack adhérent} (avec courriels), lot de services divers, de la solution e-mail complète et clé-en-main au DNS. Services basiques (courriels, jabber, etc.). Espace de départ pour les courriels de 1Go. Nombre d’adresses non limité, consultables via IMAP(S) et webmail, et utilisant un ou plusieurs noms de domaines personnels ou un ou plusieurs sous-domaines gratuits de acteurdu.net ou altu.fr.

			\item[$\bullet$] \textbf{Serveur virtuel dédié} (VPS), vous êtes maître de votre système GNU/Linux, tout en profitant du confort d’un hébergement en datacenter. Il n’y a quasiment aucune contrainte, mais vous êtes responsable de tout. Système Debian. Installation personnalisée. Espace de départ à 30Go avec 256Mo de mémoire. Livré avec un bloc IPv6 /56 et une IPv4 fixes. Possibilité d’utiliser un ou plusieurs sous-domaines gratuits de acteurdu.net ou altu.fr pour les services accessibles dessus. \\
		\end{itemize}

	\subsection{Wiki LDN}
		\vspace{0.3cm}
		Pour nous aider à réaliser le projet, notamment pour la reproduction la plus proche possible du réseau que celui qui est mis en production, les tuteurs nous ont donné un accès complet à la documentation établie par l'association (\url{https://wiki.ldn-fai.net})

\newpage
