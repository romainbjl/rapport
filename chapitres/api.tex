\section{API Zabbix}
	\subsection{Qu'est-ce que c'est ?}
		\vspace{0.3cm}

		Zabbix met à disposition un outil intéressant pour obtenir des informations sur les différents hôtes supervisés. Cet outil est une API (Application Programming Interface). L'API peut être utilisable sous différents langages de programmation, le PHP (avec la technologie Jsonrc), le Perl, le Ruby, le Python... Son fonctionnement en PHP est simple il suffit d'envoyer un script PHP par une méthode POST et le navigateur affiche la réponse dans une page web. Son fonctionnement en Perl est aussi très similaire, nous devons créer un script et de le lancer pour obtenir une réponse dans le shell. Utiliser l'API dans un langage de scripting peut être une bonne alternative si il n'y a pas de navigateur web à disposition.\\

		Cet outil, puissant, permet d'étendre les fonctionnalités de Zabbix pour le développement d'applications ou pour l'intégration avec des logiciels tiers.\\

	\subsection{Utilisation de l'API}
		\vspace{0.3cm}

		L'API de Zabbix permet d'obtenir une multitude d'informations concernant un hôte ou un composant, mais il est également possible de récupérer des informations sur les utilisateurs, leurs différentes vues sur l'application mais aussi les scripts qui leurs sont liés.\\

		Nous utilisons l'API dans l'optique d'arriver à créer un déclencheur qui avertirait l'utilisateur qu'il dépasse sa limite du 95\up{ème} centile défini. Pour cela nous avons recherché un moyen de récupérer la valeur du centile grâce aux méthodes de l'API.\\

		Nous nous sommes vite rendu compte que la valeur du centile était recalculée par Zabbix à chaque rechargement du graphique, mais n'était pas stockée dans la base de données, donc inutilisable. A moins de récupérer les valeurs du débit sur les graphiques grâce à la méthode de l'API et de recalculer le 95\up{ème} centile à la main, il était impossible de le récupérer via l'API.\\

	Voici un tableau résumant les différents points contrôlables avec l'API :

		\begin{tabular}{|c|c|m{9cm}|}
\hline
\bf Catégories & \bf Propriété & \bf Description \\

\hline
\bf Surveillance & Historique & Permet de récupérer les valeurs recueillies par les processus de surveillance de Zabbix comme la présentation de la machine surveillée et les traitements effectués. Pour obtenir ces valeurs il faut utiliser la méthode suivante : "history.get". \\

\hline
 & Événements & Récupère les évènements générés par les déclencheurs, la découverte réseau et d'autres systèmes de Zabbix. Pour la récupération des valeurs des évènements, il faut utiliser les méthodes suivantes : "event.get" et "event.acknowledge".\\

\hline
. & Services & Récupérer les données concernant les informations de disponibilité sur une machine. Pour cela il faut utiliser la méthode : "service.getsla".\\

\hline
\bf Configuration & Hôtes et groupe d'hôtes & Gérer les hôtes, les groupes d'hôtes et tout ce qui touche à leur disposition y compris les interfaces, macros et les périodes de maintenance. Des méthodes à retenir : "host.get", "host.update" et "host.massupdate". \\

\hline
. & Déclencheurs & Gérer et configurer les déclencheurs pour informer des problèmes dans le système. Permet aussi de gérer les dépendances de déclenchement. Quelques méthodes importantes : "trigger.adddependencies", "trigger.get" et "trigger.update". \\

\hline
. & Graphiques & Modifier les graphiques ou récupérer les valeurs travaillées par celui-ci. Deux méthodes importantes pour la récupération de valeur et de mise à jour des graphes : "graph.get" et "graph.update".\\

\hline
. & Templates & Gérer les templates et les relier à des hôtes ou d'autres templates. Les méthodes permettant ces actions : "template.update" et "template.massupdate".\\

\hline
. & Découverte de bas niveau & Configurer les règles de découverte de bas niveau ainsi que les déclencheurs et graphiques prototypes. Méthode pour la gestion des graphiques prototypes : "graphprototype.get" et "graphprototype.update".\\

\hline
. & Screen (vue utilisateur) & Modification global des screens ou d'un élément particulier de celui-ci. Méthode de modification globale : "screen.update". \\ 

\hline
\bf Administration & Utilisateurs & Permet l'ajout, la modification, la suppression et la mise à jour des utilisateurs de Zabbix. Méthodes importantes : "user.update", "user.login/logout", "user.addmedia" et "user.get". \\

\hline
. & Scripts & Configurer et exécuter des scripts pour aider dans les tâches de supervision. Méthode : "script.get", "script.update" et "script.execute". \\

\hline
\end{tabular}

\vspace{0.3cm}

		Un aperçu d'un script de l'API est disponible en annexe, voir point \ref{API}.\\

\newpage
