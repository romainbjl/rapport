\section{RRDTool}
	\subsection{Qu'est-ce que c'est ?}
		\vspace{0.3cm}

		RRDTool (Round-Robin Database Tool) est un outil de gestion de base de données RRD créé par Tobias Oetiker. Il est utilisé par de nombreux outils, tels que Nagios ou encore Cacti pour la sauvegarde de données et le tracé des graphiques. Cet outil est très utile pour superviser la bande passante ou la température d'un processeur. Le principal avantage d'une base RRD est sa taille fixe. \\

\begin{center}

		\includegraphics{rrd.png}

		\vspace{0.3cm}

\end{center}

		Nous avons utilisé cet outil dans l'optique de recréer les graphiques de Zabbix pour par la suite mettre en place des vues personnalisées pour chaque utilisateur.\\

	\subsection{Les vues utilisateur (Screen)}
		\vspace{0.3cm}

		Zabbix met à disposition un système de vue pour chacun de ses utilisateurs. Ce système, Screen, permet de créer un tableau de bord composé d'éléments définis par l'administrateur. Ces éléments peuvent être très variés, insérer un graphique indiquant la courbe d'évolution de l'activité du CPU de la machine de l'utilisateur par exemple.\\

		Un des objectifs du projet tuteuré étant de trouver un moyen de restreindre la vision de l'utilisateur sur Zabbix à sa seule machine, cet outil semblait remplir les différents critères.\\

		Malheureusement, le plan de notre infrastructure ne convenait pas au bon fonctionnement de cet outil. En effet, n'ayant qu'une seule machine physique, regroupant un grand nombre de machines virtuelles, nous ne pouvions installer l'agent Zabbix uniquement sur cette dernière. Il est impossible de donner accès uniquement à un seul composant (la machine virtuelle) à l'utilisateur. Ce problème s'explique par le fait que le serveur Zabbix est enregistré dans un groupe d'hôtes. Les permissions de chaque utilisateur ne peuvent être reliées qu'à des groupes d'hôtes, nous donnerons accès aux vues des autres machines virtuelles à chaque utilisateur.\\

Voici un aperçu d'un Screen de Zabbix :

\begin{center}

		\includegraphics[width=12cm]{screen.png}

		\vspace{0.3cm}

\end{center}

	
\vspace{0.3cm}


	\subsection{RRDTool}
		\vspace{0.3cm}


		\subsubsection{Requête MySQL}
			\vspace{0.3cm}

		Pour pouvoir intégrer RRDTool dans l'outil de supervision Zabbix nous avons choisi de créer un script en Perl et de l'appeler par la suite lors d'une alerte d'un déclencheur par un script lié à celui-ci.\\

		Au préalable nous avons fait quelques recherches sur la base de données de Zabbix pour localiser l'endroit où sont stockées les valeurs du débit des graphiques.\\

		Voici les quelques requêtes SQL que nous avons effectuées pour la localisation : \\

\fcolorbox{gray}{black}{
				\begin{minipage}{0.9\textwidth}
				\color{white}

 SHOW COLUMNS FROM graphs;

				\color{black}
				\end{minipage}
}
\\

Cette requête permet de retourner toutes les colonnes qui composent la table des graphiques pour nous permettre ainsi d'identifier plus facilement le contenu.\\

\fcolorbox{gray}{black}{
				\begin{minipage}{0.9\textwidth}
				\color{white}

SELECT graphid FROM graphs WHERE name='95 centile ...';

				\color{black}
				\end{minipage}
}
\\

Cette requête permet de filtrer l'identifiant du graphique du 95 centile.\\

\fcolorbox{gray}{black}{
				\begin{minipage}{0.9\textwidth}
				\color{white}

SELECT itemid FROM graphs\_items WHERE graphid = 549;

				\color{black}
				\end{minipage}
}
\\

Nous sélectionnons l'identifiant de l'item du graphique du 95 centile pour ensuite récupérer les valeurs stockées dans la table item.\\

\fcolorbox{gray}{black}{
				\begin{minipage}{0.9\textwidth}
				\color{white}

SELECT * FROM history\_uint WHERE itemid=23331;
SELECT value FROM history\_uint WHERE itemid = 23331 AND clock=(SELECT MAX(clock) FROM history\_uint WHERE itemid=23331);

				\color{black}
				\end{minipage}
}
\\

La première requête permet de voir toutes les valeurs contenues dans la table history\_uint. Cette table représente la source de toutes les données de type numérique.\\

La seconde requête permet de récupérer la valeur du débit pour le graphique avec l'identifiant d'item numéro 23331 et on applique un filtre sur cette valeur pour obtenir la dernière en date. On peut noter que la date est en timestamp unix, c'est un format de date qui représente le nombre de secondes écoulées depuis le 1\up{er} janvier 1970 à 0h00:00. Il est possible de récupérer cette valeur de date grâce à la ligne de commande suivante :\\

\fcolorbox{gray}{black}{
				\begin{minipage}{0.9\textwidth}
				\color{white}

user@localhost\# date +\%s

				\color{black}
				\end{minipage}
}
\\

		\subsubsection{RRDTool fonctionnement}
			\vspace{0.3cm}

Le fonctionnement de RRDTool se tient en trois étapes. Tout d'abord la création d'une base de données vide ainsi que la définition des configurations du graphique. Dans un second temps, une mise à jour des données des informations du graphique. Puis enfin la génération du graphique en précisant la taille de celui-ci afin que les différentes opérations s'effectuent sur les valeurs de la base de données.\\

\fcolorbox{gray}{black}{
				\begin{minipage}{0.9\textwidth}
				\color{white}

rrdtool create test.rrd --start 1393426052 DS:test:GAUGE:600:U:U RRA:AVERAGE:0.5:1:24


				\color{black}
				\end{minipage}
}
\\

Cette première ligne de commande permet de créer la base de données appelée ici "test.rrd". Nous définissons aussi une date de début pour la génération du graphique avec le paramètre "--start" ici sa valeur est de 1393426052 équivalent au 26 février 2014 à 14h47:32. Le paramètre DS signifie Data Source on lui passe en paramètre le nom de la base de données que l'on souhaite ici "test" ainsi que le type de valeur qu'elle va contenir, ici GAUGE spécifie une valeur entrée par l'utilisateur tel qu'un nombre. Le paramètre RRA tant qu'à lui définit une archive qui se compose d'un certain nombre de valeurs de données ou des statistiques pour chacune des sources de données définies par le DS.\\


\fcolorbox{gray}{black}{
				\begin{minipage}{0.9\textwidth}
				\color{white}

rrdtool update test.rrd 1297810801:0:0 1297811101:5:1 ...

				\color{black}
				\end{minipage}
}
\\

Cette seconde ligne de commande permet la mise à jour des données de la base. Nous passons en paramètre le nom de la base à cibler par la modification ainsi que les valeurs les unes après les autres. Les valeurs sont définies comme ceci : date-en-timestamp:valeur-x:valeur-y. \\

\fcolorbox{gray}{black}{
				\begin{minipage}{0.9\textwidth}
				\color{white}

rrdtool graph test.png -s 1393426052 -e 1393426947 -h 300 -w 600 -t "Graphe de test" DEF:test=test.rrd:test:AVERAGE LINE3:test\#FF0000:"TEST"


				\color{black}
				\end{minipage}
}
\\

Cette dernière ligne de commande permet la génération du graphique RRD. On définit un nom pour le graphique, la base de données à utiliser dans DEF ainsi que sa hauteur et sa largeur avec les paramètres "-h" et "-w". Le paramètre "-t" permet de définir une légende pour le graphique. \\

Vous trouverez en annexe les différents scripts que nous avons effectués pour nos tests.\\


\newpage		
