\section{Limitation du débit}
	\vspace{0.3cm}
	
	Pour que les utilisateurs puissent contrôler leurs débits, il était prévu initialement de pouvoir mettre en place un système d'alerte, suivant la valeur du 95\up{ème} centile, puis de permettre à l'utilisateur de réduire sa consommation de sa propre initiative ou automatiquement. Bien entendu, en tant que serveur de monitoring, Zabbix comprend initialement un système d'alerte automatique. Malheureusement la limite du débit n'est pas incluse à la solution, il a fallu donc s'orienter vers un outil système, \verb?tc?, puis utiliser cet outil grâce à un script depuis l'agent Zabbix.\\

	\subsection{TC}
	\vspace{0.4cm}
	
		\verb?Tc? est un programme en ligne de commande sous Linux qui permet de gérer son trafic réseau. Il est régulièrement utilisé pour l'une de ses capacités très appréciée qui est la prise en charge du QoS (Quality of Service) : c'est la gestion prioritaire de flux, qui permet également de limiter le débit réseau sur un interface. C'est cette fonctionnalité-là que nous allons utiliser.\\
	
		Le programme permet de créer des conteneurs, aussi appelés gestionnaires ou \verb?qdisc?, qui traiteront les paquets avant leur arrivée sur la carte réseau et décideront à quel moment ces paquets seront transmis sur celle ci. Dans l'exemple qui va suivre nous utiliserons les conteneurs \verb?HTB?,\verb?H?ierarchical \verb?T?oken \verb?B?ucket.\\
		
		C'est donc de cette façon que vont être traités les paquets pour brider notre carte réseau, suivant nos conteneurs et nos configurations, \verb?tc? décidera si le paquet est légitime ou non d'être transmis à la carte réseau. La grande faiblesse de cet outil est qu'il contrôle uniquement ce qui sort des interfaces. Il faut donc passer par un routeur, ou bien par des interfaces virtuelles si l'on veut effectuer le contrôle entrant et sortant.\\
	
		 Pour contrôler les flux entrants sur un interface (eth1 dans notre exemple), il faut procéder de la manière suivante : \\

		\fcolorbox{gray}{black}{
			\begin{minipage}{0.9\textwidth}
			\color{white}
			root@services \$ \#Création d'un conteneur (ou qdisc) htb sur eth1

			root@services \$ tc qdisc add dev eth1 root handle 1:0 htb default 1
	
			root@services \$ \#Création d'une classe fille au conteneur parent avec le limitation de débit, 800000 dans l'exemple qui suit, attention valeur en bit

			root@services \$ tc class add dev eth1 parent 1:0 classid 1:1 htb rate 800000
			\color{black}
			\end{minipage}
		}
		\\
	
		On peut ensuite vérifier que la classe est bien créée grâce à la commande \verb?tc -s class show dev eth1? \\

		\fcolorbox{gray}{black}{
			\begin{minipage}{0.9\textwidth}
			\color{white}
			root@services \$ tc -s class show dev eth1

			class htb 1:1 root prio 0 rate 800000bit ceil 800000bit burst 1600b cburst 1600b

			~Sent 9718 bytes 135 pkt (dropped 0, overlimits 0 requeues 0)

			~rate 1920bit 3pps backlog 0b 0p requeues 0

			~lended: 135 borrowed: 0 giants: 0

			~tokens: 197141 ctokens: 197141
			\color{black}
			\end{minipage}
		}
		\\

		Une fois que tous les tests sont ok, on peut supprimer la classe avec la commande \verb?tc class del dev eth1 parent? \verb?1:0 classid 1:1 htb rate 800000?, puis le conteneur avec \verb?tc qdisc del dev eth1 root handle 1:0 htb default? \verb?1?.\\

	\subsection{Mise en place de script dans Zabbix}
		\vspace{0.3cm}
		
		Maintenant que l'on a correctement utilisé \verb?tc? pour limiter le débit, nous pouvons tenter de le mettre en place dans un script Zabbix. Pour que l'utilisateur final interagisse le moins possible sur le serveur, on va créer un script bash qui prendra en argument l'interface réseau de la machine virtuelle ainsi que le débit désiré, en kilobit, pour la restriction de débit.\\

		\subsubsection{\label{prerequis}Prérequis}
			\vspace{0.3cm}
		
			Pour pouvoir lancer un script depuis Zabbix il faut impérativement disposer de l'agent Zabbix, au cas où il n'est pas installé se référer au point \ref{installAgent}, ainsi que le paquet \verb?sudo? installé (\verb?apt-get install -y sudo?). Ensuite vous pouvez ajouter un shell à l'utilisateur Zabbix. En effet lors de l'installation de l'agent, un utilisateur Zabbix se crée, mais avec un shell du type \verb?/bin/false?. On peut donc éditer le fichier passwd (\verb?vim /etc/passwd?), pour y ajouter un bash restreint (\verb?rbash?). Ensuite nous allons donner des droits sudo à l'utilisateur pour qu'il puisse lancer la commande \verb?tc?. Enfin on va devoir autoriser l'agent à lancer les scripts dans le fichier de configuration de l'agent, \verb?/etc/zabbix/zabbix_agentd.conf?. Pour effectuer ces modifications : \\

			\fcolorbox{gray}{black}{
				\begin{minipage}{0.9\textwidth}
				\color{white}
				root@services \$ \#Modification du passwd 

				root@services \$ vim /etc/passwd

				\color{yellow}25 \color{white} zabbix:x:107:112::/var/lib/zabbix/:/bin/rbash \\

				root@services \$ \#On donne les droits sudo pour tc 

				root@services \$ visudo

				\color{yellow}21 \color{white} zabbix ALL=NOPASSWD: /sbin/tc \\

				root@services \$ \#Ajout du droit d'utilisation de scripts par l'agent et on le redémarre

				root@services \$ vim /etc/zabbix/zabbix\_agent.conf

				\color{yellow}62 \color{white} \# Default:

				\color{yellow}63 \color{white} EnableRemoteCommands=1 \\

				root@services \$ /etc/init.d/zabbix-agent restart
				\color{black}
				\end{minipage}
			}
			\\

		\subsubsection{Création du script et lancement via Zabbix}
			\vspace{0.3cm}

			Comme dit précédemment, nous allons maintenant pouvoir créer le script en bash et le lanceur sur le serveur Zabbix. L'objectif du script était de pouvoir laisser le choix à l'utilisateur du débit désiré, en kilobit, et d'auto-compléter la partie sur le choix de l'interface réseau sur laquelle cette restriction va s'effectuer. Malheureusement, malgré de nombreuses recherches sur la documentation Zabbix, il semblerait qu'il n'était pas possible de pouvoir récupérer la valeur de l'interface réseau courante sur laquelle le déclencheur se met en marche. Nous avons donc pris la liberté de mettre en argument l'interface réseau désirée. Voici le script final qui reste donc très simple :

			\fcolorbox{gray}{black}{
				\begin{minipage}{0.9\textwidth}
				\color{white}
				root@services \$ mkdir /etc/zabbix/scripts \&\& vim /etc/zabbix/scripts/monscript.sh
				
				\color{yellow}1 \color{white} \#/bin/bash
				
				\color{yellow}2 \color{white} debit=\$((2 * 1024))
				
				\color{yellow}3 \color{white}
				
				\color{yellow}4 \color{white} sudo /sbin/tc qdisc add dev \$1 root handle 1:0 htb default 1
				
				\color{yellow}5 \color{white} sudo /sbin/tc class add dev \$1 parent 1:0 classid 1:1 htb rate \$debit \\
				
				root@services \$ chmod +x /etc/zabbix/scripts/monscript.sh
				\color{black}
				\end{minipage}
			}
			\\
			
			Une fois le script créé, nous allons définir le déclencheur automatique pour l'action pour lancer le script à partir de Zabbix.

	\subsection{Mise en place des déclencheurs automatiques}
		\vspace{0.3cm}		

		Il est possible de configurer un déclencheur qui va agir sur plusieurs unités. Par exemple une valeur de consommation limite d'un hôte, une activité/inactivité, etc.. À l'image des graphiques prototypes, il existe la possibilité de créer des déclencheurs qui se lieront automatiquement à un équipement dès sa découverte. L'image suivante est la fenêtre de création et de configuration des déclencheurs prototypes.\\

	\begin{center}
		\includegraphics[width=10cm]{declencheur1.png}
		\vspace{0.2cm}
	\end{center}

	Pour la déclaration d'un nouveau graphique, il y a à disposition six variables configurables.\\

	\begin{tabular}{|m{3cm}|c|m{10cm}|}
		\hline
		\bf Propriété & \bf Type de variable & \bf Description \\

		\hline
		Nom & string & Nom du déclencheur prototype, il est possible d'utiliser la variable \{\#IFNAME\} dans le nom du déclencheur. Cette variable permet l'affichage dynamique de l'interface réseau que l'on visualise.\\

		\hline
		Expression & - & L'expression est l'aspect qui va déclencher l'alerte. Il est possible de définir des alertes pour n'importe quelle valeur retournée par l'agent Zabbix de l'hôte et les autres protocoles. Définition des expressions (voir image suivante)\\

		\hline
		Constructeur d'expression & lien & Le constructeur est un outil pour créer des liens entre les différentes expressions sélectionnées dans la zone précédente. \\

		\hline
		Génération d'évènements problème multiples & entier & Activer ou désactiver permet l'affichage de l'ensemble des problèmes relevés par le déclencheur lors d'une alerte.\\

		\hline
		Description & string & Utilisation de ce champ pour définir l'utilisation du déclencheur ainsi que l'action apportée et la solution/explication de l'alerte.\\

		\hline
		URL & lien & Ajouter un lien vers une solution pour le problème de l'alerte.\\

		\hline
		Sévérité & entier & Six valeurs pour classer le déclencheur suivant une hiérarchie de danger : 0- Non classé (sévérité neutre, un évènement qui n'a pas de grade de danger), 1- Information (Grade informatif, le déclencheur donne une information supplémentaire sur l'action en cours), 2-Avertissement (Grade bas niveau, le danger n'est pas important mais doit être consulté et résolu), 3-Moyen (Danger modéré, le système n'est pas en danger mais pourrait le devenir), 4-Haut (Danger important, le système est peut être en danger, problème à résoudre vite), 5-Désastre (Système détérioré).\\

		\hline
		Activé & - & Active/désactive le déclencheur.\\

		\hline
	\end{tabular}

	\vspace{0.3cm}

	Les déclencheurs actifs sont indiqués dans le menu Supervision/Tableau de bord ou dans Supervision/Déclencheurs. Pour clôturer un déclencheur, il faut l'acquitter. Pour cela il est nécessaire de fournir un descriptif de la solution pour pouvoir le valider dans Zabbix.\\

	\begin{center}
		\includegraphics[width=10cm]{declencheur2.png}
		\vspace{0.3cm}
	\end{center}
	
	\begin{tabular}{|m{3cm}|c|m{10cm}|}
		\hline
		\bf Propriété & \bf Type de variable & \bf Description \\

		\hline
		Éléments & - & Ayant choisi de créer un déclencheur prototype il faut donc sélectionner un autre objet prototype pour effectuer un lien propre qui permettra la création automatique de l'ensemble. Voir image suivante pour les différents éléments disponibles. \\

		\hline
		Fonction & - & Différentes fonctions sont disponibles, parmi elles nous utilisons la suivante : La dernière (plus récente) valeur T est > N.\\

		\hline
		Dernier (T) & entier & Définition en seconde du temps d'intervalle entre chaque T pour la comparaison des deux valeurs. Il y a deux types d'intervalles disponibles, le type par défaut en seconde ou le type compte qui permet de définir le nombre de valeurs récupérées avant l'opération. Par exemple, si il est renseigné 1 dans Compte, l'opération s'effectuera à toutes les valeurs retournées.\\

		\hline
		Décalage temporel & entier & Définition du temps de décalage en seconde entre chaque T.\\

		\hline
		N & entier & Valeur de comparaison à définir.\\

		\hline
	\end{tabular}

	\vspace{0.3cm}

	\begin{center}
		\includegraphics[width=10cm]{declencheur3.png}
		\vspace{0.3cm}
	\end{center}
	
	Dans la fenêtre de choix de l'élément, on peut choisir tous les éléments disponibles sur l'hôte et sur le modèle lié au déclencheur. Ici pour notre configuration nous avons choisi Incoming network traffic on \{\#IFNAME\} et Outgoing network traffic on \{\#IFNAME\} correspondant au débit entrant et sortant de l'interface.\\

	\subsection{Les actions}
		\vspace{0.3cm}

		Une fois votre déclencheur mis en place, il va falloir créer une action pour que la restriction du débit via \verb?tc? s'effectue. Cela peut se faire très simplement depuis l'interface web de Zabbix.\\

	Pour mettre en place l'action, nous allons aller dans "configuration", puis dans "actions". Une fois dans cette page, sélectionner le bouton en haut à gauche "Créer une action". Une nouvelle page s'ouvre composée de trois onglets pour configurer votre action. Pour commencer, sélectionner l'onglet "Action" où nous allons tout simplement donner un nom, "Limite débit eth1" dans notre cas. Penser bien évidement à cocher "Activé" pour que l'action soit mise en place.\\

		Ensuite il y a "Condition". Cet onglet représente la condition que le serveur doit avoir pour pouvoir lancer l'action. Nous allons donc choisir la condition reliée au déclencheur créé précédemment qui va permettre de lancer l'action dès qu'il sera activé. Vous pouvez voir un exemple dans l'image qui suit.\\

		\begin{center}
			\includegraphics[scale=0.8]{action0.png}
		\vspace{1cm}	
		\end{center}
		
		Enfin il faut mettre l'action qui va être effectuée lorsque la condition sera remplie. Pour cela nous devons aller dans l'onglet "Opération". Dans cette partie nous pouvons configurer l'action avec les configurations qui suivent, dans cet exemple le bridage se fera sur la carte réseau eth1 pour une valeur maximum de 800Kb/s (voir l'image de l'exemple sur le point \ref{action}) :\\

		\vspace{0.5cm}

		\begin{itemize}
			\item[$\bullet$] Type d'opération : Commande à distance

			\item[$\bullet$] Liste des cibles : router

			\item[$\bullet$] Type : Script personnalisé

			\item[$\bullet$] Exécuté sur : agent Zabbix

			\item[$\bullet$] Commandes : /etc/zabbix/scripts/tc.sh eth1 800\\
		\end{itemize}
	
		Vous pouvez désormais sauvegarder et votre bridage s'effectuera au lancement de votre déclencheur.\\
			
\newpage
