\section{Introduction}
	\vspace{0.3cm}
	Dans le cadre des projets tuteurés de la licence professionnelle ASRALL, nous avons eu pour objectifs de concevoir une interface permettant d'informer en temps réel les abonnés d'un FAI à propos de leur consommation de bande passante ainsi de pouvoir leur permettre de réduire cette dernière. L'association Lorraine Data Network est un FAI (fournisseur d'accès à Internet) mais aussi un FSI (fournisseur de services Internet). Le but de l'association est d'offrir des services aux utilisateurs pour défendre sa vision du réseau.\\ 

	C'est en effet une association à but non-lucratif, pour la défense d'un Internet libre, neutre et décentralisé. La facturation du transitaire étant basée sur le débit, l'usage fait de leur ligne par les adhérents a un impact direct sur celle-ci. L'objectif est donc de pouvoir permettre aux usagers de prendre conscience de leur impact sur le réseau, en fonction des différents services souscrits.\\
	
	Nous avons donc dû recréer le réseau en production de manière minimaliste, de façon à pouvoir l'utiliser comme plateforme de test. Pour ce faire nous avons installé Debian Wheezy sur 2 machines, la première servant de routeur, et la seconde hébergeant les différents services. Il fallait ensuite une solution pour calculer les débits différenciés à partir d'une interface réseau, d'un vhost ou d'un service pour les représenter sous forme de graphiques (RRD) et indique le 95\up{ème} centile correspondant à l'abonné. Les tuteurs nous ont orientés sur la solution Zabbix. Les utilisateurs devront ensuite être en mesure de les consulter avec un accès limité (vues) sur le serveur Zabbix.\\

Le projet devra également être capable de lever des alertes à partir de seuils et éventuellement de restreindre le débit de l'interface / du vhost / du service. 

\newpage
