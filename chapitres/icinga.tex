\section{Icinga}
	\subsection{Présentation de Icinga}
		\vspace{0.3cm}
		
		Icinga est un fork de Nagios rétro-compatible. Ainsi, les configurations, les plugins et addons de Nagios peuvent tous être utilisés avec Icinga. Bien qu'Icinga conserve toutes les caractéristiques existantes de son prédécesseur, il s'appuie sur celles-ci pour ajouter de nombreux correctifs et fonctionnalités longtemps attendues et demandées par la communauté des utilisateurs.
	
	\subsection{Les différents composants d'Icinga}
		\vspace{0.3cm}
		
		\textbf{Icinga-Core :}\\
		
		Le noyau Icinga permet le suivi des tâches et la réception des résultats de vérification depuis de nombreux plugins. Il communique ensuite les résultats à IDODB via l'interface IDOMOD, puis le service IDO2DB chiffre ces données au travers du SSL. Bien que l'interface IDOMOD et le service IDO2DB (précédemment connu sous le nom IDOUtils) soient ensemble avec le noyau, se sont des composants indépendants pouvant être séparés afin de distribuer l'information à un ensemble de machines.\\

		\textbf{Icinga-Web :}\\
		
		Icinga-web est une interface web permettant de consulter les alertes de supervision ainsi que d'envoyer des commandes au noyau Icinga. Vous pouvez consulter, en temps réels, l'état des hôtes et services, l'historique des alertes, les notifications et la carte des disponibilités afin de garder un contrôle sur la santé de votre infrastructure. Il est possible d'intégrer la génération de graphiques de performances dans les rapports grâce à PNP ou GrapherV2. L'interface web, développée en Ajax, est flexible et personnalisable grâce au “glisser-déposer” de composants appelés cronks. Pour plus d'informations, vous pouvez consulter l'interface.\\

		\textbf{Icinga-Mobile : }\\
		
		Si vous avec besoin consulter l'état de votre système d'information lors de vos déplacements, Icinga Mobile est une application fonctionnant sur iPhone, Android et (à venir) Blackberry. Cette interface pour mobile permet de visualiser en détails l'état des hôtes et des services, de trier les données et d'envoyer des commandes comme dans Icinga-Web.\\

		\textbf{Icinga-Reporting :}\\
		
		La génération de rapports est primordiale dans la supervision. Icinga est donc un outil indispensable que ce soit pour consulter l'état des SLA, de l'utilisation des capacités ou tout simplement pour faciliter la compréhension des graphiques pour les managers. Icinga intègre un composant qui utilise JasperServer et iReport afin de générer des rapports visuels exportables en différents formats.\\

	\subsection{Installation d'Icinga-Core}
		\subsubsection{Installation des pré-requis}
			\vspace{0.3cm}
			Packages des prérequis : Apache, GCC, GD
			
			\fcolorbox{gray}{black}{
				\begin{minipage}{0.9\textwidth}
				\color{white}
				
				root@services \$ \#Installation des prérequis pour Incinga-Core
				
				root@services \$ apt-get install -y apache2 gcc glibc glibc-common gd gd-devel libjpeg libjpeg-devel libpng libpng-devel php-soap
				
				\color{black}
				\end{minipage}
			}
			\\
			
			
			On installe ensuite les prérequis pour la base de données, nous choisissons MySQL. Icinga fonctionne aussi avec PgSQL et Oracle.
			
			\fcolorbox{gray}{black}{
				\begin{minipage}{0.9\textwidth}
				\color{white}
				
				root@services \$ \#Installation de la base de données
				
				root@services \$ apt-get install -y mysql mysql-server libdbi libdbi-devel libdbi-drivers libdbi-dbd-mysql
				
				\color{black}
				\end{minipage}
			}
			\\

 		\subsubsection{Installation Icinga avec IDOUtils}
 			\vspace{0.3cm}
 			Sur Debian, l’utilisation du dépôt Backports permet d’installer les paquets disponibles dans les versions testing, voire unstable de Debian sur une version stable. Ce dépôt a pour avantage d’avoir des versions de paquets plus à jour que ceux disponibles sur le canal stable.\\

			Autoriser les paquets de Backports :
			
			\fcolorbox{gray}{black}{
				\begin{minipage}{0.9\textwidth}
				\color{white}
				
				root@services \$ \#Ajout du dépôt backports dans le fichier sources.list

				root@services \$ echo “deb http://backports.debian.org/debian-backports squeeze-backports main” >> /etc/apt/sources.list.d/squeeze-backports.list

				root@services \$ \#Mise à jour des fichiers disponibles dans les dépôts apt
				
				root@services \$ apt-get update
				
				root@services \$ \#Installation du client MySQL

				root@services \$ apt-get install libdbd-mysql mysql-client
				
				\color{black}
				\end{minipage}
			}
			\\
			
			Installation des paquets d’Icinga dans les sources backport :
			
			\fcolorbox{gray}{black}{
				\begin{minipage}{0.9\textwidth}
				\color{white}
				
				root@pc-asrall \$ apt-get -t squeeze-backports install icinga icinga-cgi icinga-core icinga-doc icinga-							idoutils
				
				\color{black}
				\end{minipage}
			}
			\\
			
			Étant donné qu'Icinga est un fork de Nagios, il est possible d'utiliser directement les plugins de Nagios. Nous installons donc le paquet \verb?nagios-plugin?.
			
			\fcolorbox{gray}{black}{
				\begin{minipage}{0.9\textwidth}
				\color{white}
				root@services \$ \#Installation du paquet nagios-plugin
				root@services \$ apt-get install nagios-plugins 
				\color{black}
				\end{minipage}
			}
			\\
			
		\subsubsection{Activation du module idomod}
			\vspace{0.3cm}

			Afin de proposer une solution évolutive permettant l'exploitation de données directement issues d'Icinga, nous installons le module \verb?idomod? qui fera la liaison entre Icinga et MySQL. De ce fait Icinga utilisera la base de données à la place de fichiers texte. Pour cela il suffit de copier le fichier \verb?/usr/share/doc/icinga-idoutils/examples/idoutils.cfg? \verb?-sample? vers \verb?/etc/icinga/modules/idoutils.cfg?.\\

		\subsubsection{Autoriser les commandes externes (CGI)}
			\vspace{0.3cm}
			
			Activer les commandes externes dans le fichier \verb?/etc/icinga/icinga.cfg? : \\
			\verb?check_external_commands=1? \\
			
			Mise à jour des propriétaires des fichiers :\\
			
			\fcolorbox{gray}{black}{
				\begin{minipage}{0.9\textwidth}
				\color{white}
				root@services \$ \#Arrêt du service icinga pour éviter les conflits

				root@services \$ service icinga stop

				root@services \$ \#Mise à jour des propriétaire de /var/lib/icinga avec droits en lecture et écriture pour les utilisateurs nagios et www-date

				root@services \$ dpkg-statoverride --update --add nagios www-data 2710 /var/lib/icinga/rw

				root@services \$ \#Mise à jour des propriétaires de /var/lib/icinga 

				root@services \$ dpkg-statoverride --update --add nagios nagios 751 /var/lib/icinga

				root@services \$ service icinga start
				\color{black}
				\end{minipage}
			}
			\\
			
		\subsubsection{Installation de Icinga Web}
			\vspace{0.3cm}
			
			On installe ensuite les paquets prérequis pour le fonctionnement de l'application web d'Icinga.\\
			
			\fcolorbox{gray}{black}{
				\begin{minipage}{0.9\textwidth}
				\color{white}
				
				root@services \$ \#Installation du de l'application web d'Incinga

				root@services \$ apt-get install php5 php5-cli php-pear php5-xmlrpc php5-xsl php5-gd php5-ldap php5-mysql 
				
				\color{black}
				\end{minipage}
			}
			\\
			
			On peut ensuite installer l'application web qui va nous permettre de configurer Icinga et de consulter les données recueillies.\\
			
			\fcolorbox{gray}{black}{
				\begin{minipage}{0.9\textwidth}
				\color{white}
				
				root@services \$ \#Installation de l'application web Icinga

				root@services \$ apt-get -t squeeze-backports install icinga-web
				\color{black}
				\end{minipage}
			}
			\\
			
			Icinga Web est accessible depuis votre navigateur Web :\\
			
			\begin{center}
			 \includegraphics[scale=0.45]{icinga1.png} 
			 \end{center}
			 
	\subsection{Addons officiels}
		
		Les addons officiels sont des programmes fournis par nagios.org pour améliorer et étendre les fonctionnalités de Nagios et de ses forks.\\
	
			\subsubsection{NRPE :}
				\vspace{0.3cm}
		
				NRPE (Nagios Remote Plugin Executor) est un agent de supervision qui vous permet de récupérer les informations à distance. Son principe de fonctionnement est simple : il suffit d’installer le démon sur la machine distante et de l’interroger à partir du serveur Icinga.\\
		
				\begin{center}
					\includegraphics[scale=0.65]{icinga2.png} 
				\end{center}
		
					Il est défini comme l’agent d’interrogation de type actif car c’est le serveur Icinga qui va interroger la machine distante.
		
			\subsubsection{NDOUtils :}
				\vspace{0.3cm}
		
				NDOUtils est un addon servant à injecter les informations de Nagios dans la base de données MySQL. Ceci permet de ne plus avoir l’ancienne gestion des archives via fichiers logs et offre donc une plus grande souplesse dans la manipulation des données. Cet addon a permis d’avoir une plus grande ouverture sur l’exploitation des résultats de Icinga et de transformer l’information de la manière que l’on souhaite.	\\
			
				\begin{center}
					\includegraphics[scale=0.65]{icinga3.png} 
				\end{center}
		
	\subsection{Comparatif Icinga / Zabbix :}
\begin{tabular}{|p{3cm}|p{7cm}|p{7cm}|}
\hline
 & \bf Icinga & \bf Zabbix \\
\hline
\bf Fonctionnalités & -Offre une interface web basée sur les CGL avec gestion des droits pour la consultation.\newline 
				- Génère des rapports de surveillance (SLA) et des documents qui définissent la qualité de service . \newline
				- Permet de surveiller à distance à travers un pare-feu. \newline
				- Permet de définir des serveurs esclaves qui prennent le relais si le serveur maitre tombe en panne. \newline
				- Surveille les ressources des serveurs (CPU, mémoire…) \newline
				- Surveille les services réseaux (http ,ssh , DNS ..) . \newline
				- Permet l'arrêt temporaire de la supervision locale ou globale. \newline
				- Génère des graphes par l'interface avec RRDTools.
				& 
				- Monitoring : La partie affichage des statistiques, graphiques, alertes, cartographie..etc.. \newline
				- Inventory : l'inventaire des machines et équipements . \newline
  				- Report : Statistiques sur le serveur Zabbix et rapport de disponibilité des services sur les machines supervisées . \newline
				- Configuration : Comme son nom l'indique, permet de configurer entièrement Zabbix \newline 
				- Administration : Permet de gérer les moyens d'alertes (SMS, Jabber, Email, ...) et les utilisateurs . \\
\hline 
\bf Architecture &
 
				Architecture généralement basée sur le moteur de l'application (écrit en C) qui sert à ordonnancer les tâches de supervision et une interface web réalisée à l'aide des CGI, décrivant la vue d'ensemble sur système et les anomalies possibles; ainsi que plusieurs plugins qui peuvent être complétés en fonction des besoins.
				&
				Architecture de  Zabbix est basée sur le cœur du moteur de l'application qui est programmé en C Le « serveur ZABBIX » peut être décomposé en 3 parties séparées : \newline
				- Le serveur de données utilise MySQL, PostgreSQL ou Oracle pour stocker les données \newline
				- L'interface de gestion est une interface web écrite en PHP. Elle agit directement sur les informations stockées dans la base de données. \newline
				- L’utilisation de Zabbix Agent permet une meilleure surveillance des hôtes, et donc une supervision plus accrue.\\
\hline
\bf Avantages & 
				- Reconnu auprès des entreprises, grande communauté \newline
				- Énormément de plugins qui permettent d'étendre les possibilités (agents comme zabbix, reporting amélioré, etc...)\newline
  				- Une solution complète permettant le reporting, la gestion de panne et d'alarmes, gestion utilisateurs, ainsi que la cartographie du réseaux. \newline
  				- Beaucoup de documentations sur le web . \newline
				- Performances du moteur . \newline 
			  &
			  	- Une solution très complète : cartographie de réseaux, gestion poussée d'alarmes via SMS, Jabber ou Email, gestion des utilisateurs, gestion de pannes, statistiques et reporting . \newline
 				- Une entreprise qui pousse le développement, et une communauté croissante . \newline
				- Une interface vaste mais claire . \newline
				- Une gestion des templates poussée, avec import/export xml, modifications via l'interface . \newline
 				- Compatible avec MySQL, PostgreSQL, Oracle, SQLite \\
\hline
				


\end{tabular}
\newpage
